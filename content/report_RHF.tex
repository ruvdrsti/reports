\section{Introduction}
\label{sec:intro}
This report will discuss the notebook we made on restricted Hartree-Fock theory.
 We will walk trough the notebook step by step and discuss what we see there. 
 The same titles will be used, so you can follow along easily. 
 Relevant parts of the code will be displayed.

 \section{Theoretical Approach}
 \label{sec:theory}
 Before we actually dive into the code, we will discuss the theory behind it.
 The actual goal of this experiment is to calculate the energy of a molecule,
 mainly to solve the Schrödinger equation for that molecule. This, unfortunatly,
 is impossible. So we have to find an alternative. This alternative would be
Hartree-Fock theory. Bluntly put we can define another operator, the Fock operator,
that consists of one-electron operators and two-electron operators. This is 
displayed in Equation \ref{eq: fock matrix}. 

\begin{equation}\label{eq: fock matrix}
    \hat{f}(1) = \hat{h}(1) + 2\sum_i^{N/2}\hat{J}_i - \sum_i^{N/2}\hat{K}_i
\end{equation}
 J is the Coulomb operator, it accounts for the interelectronic repulsion. K is 
 exchange operator, that accounts for the stabilizing interaction between 
 electrons that have the same spin. Please do keep in mind that Equation
 \ref{eq: fock matrix} only holds for restricted closed shell systems, where the
 amount of alpha-electrons equals the amount of beta-electrons. Now let us look
 at the way these operators are defined. This is shown in Equations

 \begin{equation}\label{eq:coulomb}
    \hat{J}_i\chi_a(x_1) = \int\chi_i^*(x_j)\frac{1}{|\boldsymbol{r}_1 - \boldsymbol{r}_j|}\chi_i(x_j)dx_j\chi_a(x_1)
 \end{equation}

 \begin{equation}\label{eq:exchange}
     \hat{K}_i\chi_a(x_1) = \int\chi_i^*(x_j)\frac{1}{|\boldsymbol{r}_1 - \boldsymbol{r}_j|}\chi_a(x_j)dx_j\chi_i(x_1)
 \end{equation}
The point here is the operators themselves depend on the orbitals. So we would
need to know the orbitals to define our operator. It seems we are stuck again.
However we can apply some mathemathical tricks to get us out of this impasse. 
First we can assert that every function can be expressed as a linear combination
of basis functions. We also know that the eigenfunctions of a Hermitian operator
from such a set of basis functions. We begin by selecting a convenient basis set.
This could be a set of atomic orbitals. Then we would solve the eigenproblem in
Equation \ref{eq:eigenproblem}.

\begin{equation}\label{eq:eigenproblem}
    \boldsymbol{\hat{H}}^{core}\boldsymbol{C} = \boldsymbol{SC}c
\end{equation} 
Calculating the eigenvalues and eigenfucntions for the core Hamiltonian is 
perfectly possible, since it is a one-electron operator. We now have a set of 
orbitals which we can use to build our Fock operator. For this Fock operator, 
we can solve the generalised eigenproblem in the form of the Roothaan-Hall 
equations (see Equation \ref{eq:Roothaan-Hall}).

\begin{equation}\label{eq:Roothaan-Hall}
    \boldsymbol{FC} = \boldsymbol{SC}\epsilon
\end{equation}

Solving this equation will give us a set of eigenfunctions of the Fock matrix. 
These functions can then be used to build a new Fock matrix. This is where we 
start the iterations. We can continue untill we reach a point where the energy 
does not change any more. At that point, we have found the energy of the molecule.

At this point we still need to adress one topic, the density matrix. This will be
very important since it will be used to calculate the Fock matrix in our program.
To demonstrate what it is, we will look at the expression for the coulomb operator.
When we take Equation \ref{eq:coulomb} and then expand the functions $\chi_i$ in
a basis we find Equation \ref{eq:coulombexp}.

\begin{equation}\label{eq:coulombexp}
    \hat{J}_i = \int\sum_{\sigma}C^*_{\sigma}\phi^*_{\sigma}(x_j)\frac{1}{|\boldsymbol{r}_1 - \boldsymbol{r}_j|}\sum_{\nu}C_{\nu}\phi_{\nu}(x_j)dx_j
\end{equation}

The C factors in this equation are merely the exapansion coëficcients of the function
$\chi_i$. These are constans, so we can safely extract them from the integrals to form
Equation \ref{eq:coulombexp1}.

\begin{equation}\label{eq:coulombexp1}
    \hat{J}_i = \sum_{\sigma}\sum_{\nu}C_{\nu j}C^*_{\sigma j}\int\phi^*_{\sigma}(x_j)\frac{1}{|\boldsymbol{r}_1 - \boldsymbol{r}_j|}\phi_{\nu}(x_j)dx_j
\end{equation}
We can apply this same strategy to the exchange operator. In the end we will be
able to factor out the products of the expansion coëficcients and afer summation
over all basis functions we can define the density matrix as Equation 
\ref{eq:densitymatrix}.

\begin{equation}\label{eq:densitymatrix}
    D_{\sigma\nu} = 2\sum^{N/2}_jC_{\sigma j}^*C_{\nu j}
\end{equation}
This will be used to calculate the Fock matrix. (see below)
\section{Identifying the Molecule}
\label{sec:step1}
In this first step we initiated the class molecule, which will be equipped with 
the necessary methods to do all the calculations we will be required to do. 
A molecule object has several properties that need to be defined first. 


\begin{python}[caption={intitialising the molecule object},label={ls:Listing 1}]
    class molecule:
        def __init__(self, geom_file):
            if """pubchem""" in geom_file:
                self.id = psi4.geometry(geom_file)
            else:
                self.id = psi4.geometry(f"""
                {geom_file}
            
                units bohr
                """)
            self.id.update_geometry()
            self.wfn =  psi4.core.Wavefunction.build(self.id, 
                            psi4.core.get_global_option('basis'))
            self.basis = self.wfn.basisset()
            self.integrals = psi4.core.MintsHelper(self.basis)
            # only works for closed shell systems
            self.occupied = self.wfn.nalpha()  
            self.guessMatrix = "empty"
    
\end{python}
  

In Listing \ref{ls:Listing 1} we the \pythoninline{__innit__} method of the 
molecule class. It sets us up with some of the information we will need later, 
namely the \pythoninline{psi4.core.Molecule} representation as 
\pythoninline{self.id}. We also see the wave function, basis set, 
integrals, occupied orbitals and a guess matrix. 
Since we will be doing Hartree-Fock calculations, which involve a 
lot of iterations, the molecule object will need a way to store the Fock 
matrix from the last iteration. From there we can then start for the next 
iteration. Hence, the first method we actually have to define is a method that 
allows us to change this parameter.


\begin{python}[caption={setting the guessMatrix},label={ls:Listing 2}]
    def setGuess(self, new_guess):
        """
        sets the guessMatrix to a new value

        input:
        new_guess: numpy array that represents a new fock matrix
        """
        self.guessMatrix = new_guess
\end{python}

This is shown in Listing \ref{ls:Listing 2}. 

\section{Prerequisite Calculations}
\label{sec:step2}

In this section, we do some prerequisite calculations. These are relatively 
straightforward. In this step, we added some methods to the class that allow us to 
calculate important properties like the nuclear repulsion kinetic energy and so 
on. The commands used are directly implemented from the psi4 package. 
They are listed in Table \ref{tab:commands}

\begin{table}[hp]
    \centering
    \begin{tabular}{c|c}
        command & property \\
        \hline
        \pythoninline{self.id.nuclear_repulsion_energy()}| & nuclear repulsion energy \\
        \pythoninline{self.integrals.ao_overlap().np} & overlap matrix \\
        \pythoninline{self.integrals.ao_kinetic().np} & kinetic energy \\
        \pythoninline{self.integrals.ao_potential().np} & potential energy \\
        \pythoninline{self.displayE_kin() + self.displayE_pot()} & the core hamiltonian \\
        \pythoninline{self.integrals.ao_eri().np} & repulsion between electrons \\
    \end{tabular}
    \caption{Commands useed to calculate various properties}
    \label{tab:commands}
\end{table}
We will not list the code for the various methods here, however when they appear
 in later blocks of code we will mention them.

\section{The inital (Guess) Density Matrix}
\label{sec:step3}
We will skip ahead to the function that gives us the density matrix and start
 from there.

 
\begin{python}[caption={calculating the density matrix},label={ls:Listing 3}]
        class molecule(molecule):
            def getDensityMatrix(self):
                """
                generates the densitiy matrix on the AO level
                """
                C = self.getEigenStuff()[1]
                A = 2*np.einsum("pq, qr->pr", C[:, :self.occupied], 
                                C[:, :self.occupied].T, optimize=True)
                return A
\end{python}

This function uses the method \pythoninline{getEigenStuff}, which just calls 
the scipy function \pythoninline{linalg.eigh}. This solves the generalised 
eigenproblem posed by the Roothaan-Hall equations \eqref{eq:Roothaan-Hall}, 
for whatever matrix that is currently in the \pythoninline{self.guessMatrix} 
parameter. C in Listing \ref{ls:Listing 3} then refers to \textbf{C} in Equation 
\eqref{eq:Roothaan-Hall}. The first matrix for which we calculate the eigenvalues 
and eigenvectors is the core Hamiltonian, so this will have to be stored as the 
\pythoninline{self.guessMatrix} before proceeding. Now we are set to build a Fock 
matrix.

\section{Updating the Fock Matrix}
\label{sec:step4}


    
\begin{python}[caption={calculating the Fock matrix},label={ls:Listing 4}]
    class molecule(molecule):
        def displayFockMatrix(self):
            """Will display the Fock matrix"""
            coulomb = np.einsum("nopq,pq->no", 
                    self.displayElectronRepulsion(), 
                    self.getDensityMatrix(), optimize=True)
            exchange = np.einsum("npoq,pq->no", 
                    self.displayElectronRepulsion(), 
                    self.getDensityMatrix(), optimize=True)
            self.fockMatrix = self.displayHamiltonian() 
                            + coulomb - 0.5*exchange
            return self.fockMatrix
\end{python}
 
 
We see that the Fock-matrix uses some the density matrix and the electronic 
repulsion matrix and the Hamiltonian , which the molecule object calls with the 
methods seen in Section \ref{sec:step2}. Since the density matrix depends on the 
current \pythoninline{self.guessMatrix}, the Fock matrix will also depend on it. 
We will discuss the importance of this in \ref{subsec: iteration}. 

 
 \section{The SCF Energy}
 \label{sec:step5}
 From the Fock matrix we derived in the previous section, we can now calculate the electronic energy acoording to Equation \eqref{eq:energy}.
 
 \begin{equation} \label{eq:energy}
     E_{elek} = \frac{1}{2}\sum_{\mu\nu}D_{\mu\nu}\cdot(H_{\mu\nu} + F_{\mu\nu})
 \end{equation}
 In python this looks like Listing \ref{ls:Listing 5}.
 
 
    
\begin{python}[caption={calculating the energy},label={ls:Listing 5}]
    class molecule(molecule):
        def getElectronicEnergy(self):
            """
            calculates the energy with the current fock matrix
            """
            sumMatrix = self.displayHamiltonian() 
                    + self.displayFockMatrix()
            return 0.5*np.einsum("pq,pq->", sumMatrix, 
                            self.getDensityMatrix())
\end{python}

The total energy is merely the sum of this electronic energy and the nuclear 
repulsion energy. We have a method for the latter defined in Section 
\ref{sec:step2}.
 
 \section{Test for Convergence}
 \label{sec:step6}
Now we only need to bring this all together to get the final energy. 
This is done in the function \pythoninline{iterator} as seen in Listing 
\ref{ls:Listing 6}.

 
   
\begin{python}[caption={iteration sequence},label={ls:Listing 6}]
    def iterator(target_molecule):
        """
        Function that performs the Hartree-Fock iterative calculations 
        for the given molecule.
        
        input:
        target_molecule: a molecule object from the class molecule
        """
        # setting up entry parameters for the while loop
        E_new = 0  
        E_old = 0
        d_old = target_molecule.getDensityMatrix()
        convergence = False
        E_list = []

        # step 2: start iterating
        itercount = 0
        while not convergence and itercount < 50:

            # calculating block: calculates energies
            E_new = target_molecule.getElectronicEnergy()
            E_total = target_molecule.getTotalEnergy()

            # generating block: generates new matrices
            F_n =  target_molecule.displayFockMatrix()
            target_molecule.setGuess(F_n)
            d_new = target_molecule.getDensityMatrix()

            # comparing block: "Are we there yet?"
            rms_D = np.einsum("pq->", np.sqrt((d_old - d_new)**2))
            if abs(E_old - E_new) < 1e-6 and rms_D < 1e-4:
                convergence = True


            # maintenance block: keeps everything going
            print(f"""iteration: {itercount}, E_tot: {E_total: .8f}, 
                        E_elek: {E_new: .8f}, 
                        deltaE: {E_new - E_old: .8f}, 
                        rmsD: {rms_D: .8f}""")
            E_old = E_new
            d_old = d_new
            E_list.append(E_new)
            itercount += 1
        
        return E_list
\end{python}

First we need to set up the parameters before we start iterating. That is done in 
the first block of code. Then we start a while loop, that will stop iterating once
we have reached convergence, or when we have reached the maximum amount of 
iterations. Inside the loop, we see four blocks of code. The calculating block 
provides us with the energy of the molecule for the given 
\pythoninline{self.guessMatrix}. The next block will generate new matrices for 
the next iterative step. In the comparing block we check the conditions for 
convergence. For this we need four values, the old and new energies and the old 
and new distance matrices. For more information, see Subsection 
\ref{subsec:step6.1}. For now we will continue to walk trough the code. 
After the comparing block we move to the final block, which sets us up for the 
next iteration. This iterations energy is stored, as well as the density matrix. 
We print out a line that summarises all the relevant values for this iteration. 
We can then move on to the next iterative step until the while loop finds one of 
its conditions is no longer met.

\subsection{On Iterations and the Molecule Object}
\label{subsec: iteration}
In this subsection we will discuss what the iteration actually does to the 
molecule object allong the way. We call a lot of methods on the object, but those 
do not change any of the fundamental properties of the object. Now pay special
attention to the generating block, where we change the 
\pythoninline{self.guessMatrix} to a new value. After this is done, all new values
that are calculated will be based on the new matrix. This is the only value that 
actually changes, but is has far reaching consequences on all the methods. 

 
 \subsection{On Comparing Values}
 \label{subsec:step6.1}
In Listing \ref{ls:Listing 6} we can see that there is a difference in the
citeria for the density matrix and the energy. However when we look at the 
equation we see that the energy uses the density matrix. Indeed we see in 
Equation \ref{eq:energy} that the energy is calculated using a 
product of the Fock matrix with the density matrix, which already uses the density
matrix as can be seen in Equation \ref{eq: fock matrix}.
 
 \begin{equation} \label{eq: fock matrix}
     F_{\mu\nu} = H_{\mu\nu} + \sum^{AO}_{\lambda\sigma}D_{\lambda\sigma}[(\mu\nu|\lambda\sigma) - \frac{1}{2}(\mu\lambda|\nu\sigma)]
 \end{equation}

Considering that the root-mean-square of the density matrix uses all elements of 
this matrix, we could say that the criterion for the density matrix is inherently
the strictest, since the energy criterion only uses one single value and the 
density matrix criterion uses a multitude of values that have to behave 
accordingly. However, this expalanation is to simple. Since the energy uses 
the density matrix, these conditions are in fact connected. In this case that 
would mean that enforcing one condition is enough to enforce the other one. 
We can indeed see that this is the case in Section \ref{sec: examples}. 

 
 \section{Some examples}
 \label{sec: examples}
 In this section, we will display some of the results from our calculations for 
 two example systems, water and methane. We will give a display of the output as 
 generated by the functions we discussed above.
 
 \subsection{water}
 \label{subsec:water}

\begin{python}[caption={iterations for water},label={ls:Listing 7},basicstyle=\scriptsize]
0, E_tot: -73.28579642, E_elek: -81.28816348, deltaE: -81.2881634, rmsD:  14.05298222
1, E_tot: -74.82812538, E_elek: -82.83049244, deltaE: -1.54232896, rmsD:  3.17285816 
2, E_tot: -74.93548800, E_elek: -82.93785506, deltaE: -0.10736262, rmsD:  0.65858574 
3, E_tot: -74.94147774, E_elek: -82.94384480, deltaE: -0.00598974, rmsD:  0.24093051 
4, E_tot: -74.94197200, E_elek: -82.94433906, deltaE: -0.00049425, rmsD:  0.08612099
5, E_tot: -74.94205606, E_elek: -82.94442312, deltaE: -0.00008407, rmsD:  0.04061885
6, E_tot: -74.94207442, E_elek: -82.94444148, deltaE: -0.00001836, rmsD:  0.01827516
7, E_tot: -74.94207865, E_elek: -82.94444571, deltaE: -0.00000423, rmsD:  0.00892060
8, E_tot: -74.94207963, E_elek: -82.94444669, deltaE: -0.00000098, rmsD:  0.00425406
9, E_tot: -74.94207986, E_elek: -82.94444692, deltaE: -0.00000023, rmsD:  0.00205358
10, E_tot: -74.94207991, E_elek: -82.94444697, deltaE: -0.00000005, rmsD:  0.00098889
11, E_tot: -74.94207992, E_elek: -82.94444699, deltaE: -0.00000001, rmsD:  0.00047710
12, E_tot: -74.94207993, E_elek: -82.94444699, deltaE: -0.00000000, rmsD:  0.00023011
13, E_tot: -74.94207993, E_elek: -82.94444699, deltaE: -0.00000000, rmsD:  0.00011102
14, E_tot: -74.94207993, E_elek: -82.94444699, deltaE: -0.00000000, rmsD:  0.00005356
\end{python}
Here we see the the amount of iterations before the convergence is reached and 
all parameters calculated during that iteration. Furtermore we can use the 
\pythoninline{oeprop} method to get the dipole and nuclear charges. The dipole 
moment is given in e\AA, the charge is given in e, where e is the elemental 
charge.

\begin{table}[ht]
    \centering
    \begin{tabular}{c|c}
         total dipole moment & 0.6034  \\
         \hline
         nuclear charges &  \\ 
         \hline
         O & -0.25302 \\
         H & 0.12651 \\
         H & 0.12651 \\
    \end{tabular}
    \caption{Some properties of water}
    \label{tab:number2}
\end{table}

\begin{figure}
    \centering
    \includegraphics[width=0.5\textwidth]{content/Capture.PNG}
    \caption{convergence for water}
    \label{fig:convergence1}

\end{figure}
\subsection{methane}
\label{subsec:methane}
 
\begin{python}[caption={iterations for methane},label={ls:Listing 8},basicstyle=\scriptsize]
0, E_tot: -36.08344857, E_elek: -49.58075304, deltaE: -49.58075304, rmsD:  35.9980315
1, E_tot: -39.56451342, E_elek: -53.06181788, deltaE: -3.48106485, rmsD:  4.73662689
2, E_tot: -39.72183632, E_elek: -53.21914079, deltaE: -0.15732290, rmsD:  0.79654919
3, E_tot: -39.72669300, E_elek: -53.22399746, deltaE: -0.00485667, rmsD:  0.14041648
4, E_tot: -39.72684535, E_elek: -53.22414982, deltaE: -0.00015236, rmsD:  0.02394664
5, E_tot: -39.72685016, E_elek: -53.22415462, deltaE: -0.00000480, rmsD:  0.00443984
6, E_tot: -39.72685031, E_elek: -53.22415477, deltaE: -0.00000015, rmsD:  0.00072904
7, E_tot: -39.72685032, E_elek: -53.22415478, deltaE: -0.00000001, rmsD:  0.00014205
8, E_tot: -39.72685032, E_elek: -53.22415478, deltaE: -0.00000000, rmsD:  0.00002652
\end{python}
 
 \begin{table}[ht]
    \centering
    \begin{tabular}{c|c}
         total dipole moment & 0.0000  \\
         \hline
         nuclear charges &  \\ 
         \hline
         C & -0.26031 \\
         H & 0.06508 \\
         H & 0.06508 \\
         H & 0.06508\\
         H & 0.06508 \\
    \end{tabular}
    \caption{Some properties of methane}
    \label{tab:number2}
\end{table}


